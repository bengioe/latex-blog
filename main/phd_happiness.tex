\title{On happiness, meaning, PhD, and a bit of metaphysics}

<h4>On happiness, meaning, PhD, and a bit of metaphysics</h4>

As I write these words I will soon begin my 5th year as a PhD student in computer science. Saying the past 4 years haven't been easy would be an understatement. At the same time, I can confidently say that I would do it all over again. Would you have ever asked me “are you happy?” I would probably have started my answer with a sigh or an unpleasant moan, yet this adventure is one of the experiences that have brought the most meaning to my life. 

But what do I mean by happy? By meaning? I will attempt here, as so many have before me, to think about this and transform in words what goes on in my head. I do so in the hope that people who know me get to know me a bit better, that other researchers reading this learn a thing or two, and perhaps also so you can teach me a thing or two (angry emails welcome).

I don’t know that I really have objectives in life, at least not in the sense that was taught to me. I certainly want things, experiences, friendships, a partner. I want to be loved, and respected for the work I do (and I mean “work” in the largest sense rather than “job”). I assume those are fairly basic wants in humans. That being said I don’t have a clear list I could write down. 

As a PhD student, one would assume that “write a thesis and graduate” would be an easy item to add to such a list, but that’s not really what I’m after. I think the most basic state of mind that I want to reproduce over and over through research is that of exploration, discovery, and understanding. Like many of my peers I started fooling around computers and code fairly young, and I don’t think anything (intellectual) beats the feeling of understanding how computers worked, compilers, systems, networks, game engines, eventually machine learning. I still crave that feeling, and I do get it occasionally now as I read and consume knowledge in my field (and on occasion when I’m the one producing that knowledge!)

So if anything, graduating brings me anxiety. What if this is the only moment in my life where I get to do this?

Here’s another feeling I greatly enjoy, teaching and helping others. I don’t know if it’s out of empathy, of me wanting others to feel the same way I do about knowledge. Maybe I just want to feel superior, ha!

Taking a step back, perhaps I don’t really have written-down objectives, but I can certainly induce objectives that my default mode has. The follow up question to that is, do those make me happy and give me meaning?

I hold the view that Happiness is complicated but ultimately understanding it is a scientific endeavour. We can go and look at what components of life are usually associated with happiness in research. As a privileged STEM student all my basic needs are certainly met; I have money, an apartment, I can afford restaurants and travel. I’m healthy, I’m a tall white cis-male and I certainly feel safe. Going up in the ladder, I have good friends I can see (well, at least interact with, thanks covid) regularly, I have a few friendships I’d qualify as “deep” that bring me joy. I have a wonderful partner (hi :). I’m not a star but I’m respected well enough among my peers (as far as I can tell of course, perhaps my status and privilege prevents information about the contrary from reaching me). Finally, I get to be creative with very little supervision, every day, and I’m even paid for it! 

And yet… there is this gnawing feeling in me of distress, sadness perhaps, I don’t know how to call it. It certainly has two major components related to research, the first one is that research is hard, and most of my time is spent failing. This I can deal with. The second component is that I get to ask “why am I doing this?” over and over. A natural skill of researchers is to question if they should be spending time on something. We are also (ideally) good at breaking down problems into subproblems. The combination of these two, as you can imagine, leads to the inevitable “why am I doing anything at all?” What’s the point?

A parenthesis on privilege: Am I at such a height of privilege that this is the main thing I find to be sad about? Perhaps. I recognize that not all PhD students are as privileged as I am, especially in other countries and disciplines; I nonetheless cannot help but judge my peers sometimes complaining with passion about what I’d consider trivialities. It is an interesting question to wonder if we get to be “justifiably” sad about the meaning of life because we have so much free thinking time and get to explore fundamental truths about the human condition, or if we are simply looking for things to be sad about because such is human nature, and these things we find are really insultingly unimportant. Don’t get me wrong, the world at large and the continual fight for social justice we must undertake are far from bringing me only joy, and don’t get me started on capitalism. I still cannot escape seeing this existential anxiety as the root of my thoughts.

Coming back to “why”, and ignoring for a moment answers like “personal fame”, “money” and all the derivatives of these answers, two ethics-based answers emerge. The first one has to do with loosely improving the human condition, the second one is a bit darker, in my opinion, and has to do with some kind of metaphysical beauty of knowledge, related to what I’ve already alluded to above, the more “raw” emotion of discovery. In these I can certainly derive intrinsic meaning, understanding the purpose of my actions within my immediate surroundings, and perhaps also extrinsic meaning, understanding the purpose of my actions on a historical scale. I’ll get to this later.

Coming back to ethics, I think I (and perhaps a lot of us) sometimes see myself as going through the emotional “pain” of the research endeavour--the endless string of failures, the social pressure to perform, the rejections by reviewers, etc.--in order to improve the human condition, loosely, by increasing the total amount of human knowledge (a task for which we barely receive any training). Ironically this may come with a grandiose sense of duty that is quite detached from reality, an unfortunate trait which I’ve observed in a concerning number of  AI/Machine Learning folks.

But it’s easy to feel terrible about the above. First of all, most knowledge created in academia is only ever indirectly useful, usually after many levels of indirection. Few of us are expanding knowledge in such concrete ways as to enhance the human condition, as it is usually agreed upon, by reducing pain, suffering, tackling healthcare or climate change or social justice. Those of us who are are probably already feeling terrible because they are faced daily with suffering (and they deserve our praise). We can go further. Most knowledge that is used “in practice” usually benefits the current system of values in which we live. We have not reached the end of history (a perhaps controversial statement among democratic liberals), who knows how our descendants will judge us? For eating meat? For burning coal? For targeted advertising? For perpetuating capitalism? For improving machine learning algorithms?

For inventing AGI? 

As a moral constructivist I see no reason to think humanity has converged in any way on ethics and moral judgement, nor that it ever will. This makes judging the usefulness (i.e. meaning) of what we work on, in the long term, a weird thought experiment.

We can go even further. I think too much I guess, so I pass this threshold all the time.

My materialistic mind tells me that humans will eventually evolve, self-destruct or whatever form Life takes eventually disappear through the heat death of the universe. It’s rough to find meaning in this. If you believe in \href{conformal cyclic cosmology}{https://www.youtube.com/watch?v=PC2JOQ7z5L0} you may be tempted to have Life endeavour to have \href{some kind of causal effect on the next universe}{https://www.youtube.com/watch?v=3N5lgUgAQ-g} and derive some epsilon-meaning in your actions through this.

I don’t think that’s a very good bet, although I am occasionally inclined to reassure myself with its possibility. 

Perhaps it is better to take a hint from Camus and assume the absurdity of our existence. I think there’s something special in knowledge though. Something quite concretely beautiful. Perhaps we can listen to Carl Sagan and Bill Hicks and accept that we are the universe experiencing itself, knowing itself. Perhaps I am wrong, and there are some grander things to understand about the universe and that \href{no one fucking knows yet}{https://www.youtube.com/watch?v=uBinqZfhIBg}. I could go on, metaphysical ideas abound.


Right. If you made it this far in my text you are probably either (1) judging me for wasting all this time thinking about inconsequential things (2) recognizing yourself in letting your mind wander and allowing it to destroy all meaning through thought.

I’m a PhD student. Thinking is most of what I do. I often just stand up, look out the window and stare. I know I’m not the only one. Thinking often makes me sad, but sometimes I have a thought I’d never had before. A thought no one ever had before. When that happens, I feel incredibly lucky, and even though I’ll eventually forget this thought, even though the paper on which I write this thought will eventually decay, the bits of information disappear on some faulty storage unit, even then, I’ll still have felt lucky.

Perhaps we should recognize PhD students for what they are, artists. Vice versa, perhaps we should give as much importance to artists as we do STEM students. For a brief moment in time, we expand human experience, we create stuff that’s never been created before. Sometimes in a stroke of luck we get to touch the lives of countless others. For better or for worse, it’s not clear. It might even be impossible to judge. 

One thing is sure, being on this endeavour occasionally makes me happy, and even though I can think myself out of having any extrinsic meaning, in a magical flash of circular logic research provides me with intrinsic meaning, and somber serenity.
