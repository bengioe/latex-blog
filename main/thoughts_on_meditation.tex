\title{Thoughts on Meditation}

I cannot claim to be an expert on meditation by any stretch of the imagination, but, I have been meditating on and off for various lengths of time since around 2009.

Here are a few thoughts. I will start by briefly describing what I do, which I will afterwards break down, as I assume some if not most of what I will write may only make sense to me if I don't explain it more carefully.


\subsection{How I meditate}

I usually sit, on a chair or on my knees, eyes closed. I sometimes stand or meditate while doing a fairly routine activity, cooking, dishes, walking.

For me the first step is to observe my body, in such a way that I become its observer and that its sensations appear to be someone else's.

The second step is to observe my thoughts as they arrive, without judgement, in such a way that when they arrive I can choose to follow them, or not. I eventually become the observer of this generative process of thoughts.

The last step is more variable, and seems to depend on my mood or inclinations. I often simply go back to my body or to observing my thoughts. Sometimes I don't really need to do anything, and for the lack of better words, I simply enjoy having calmed the river, like going from rapids to a lake.

At that stage, it's also possible to exercise various things, to ``force'' oneself to feel love, anguish, fear, hate, joy, to recall memories, events, ambitions, but always as an observer. When the observer fails, and it does, I take a step back.

\subsection{What do you mean?}

If you have never meditated, or have learned meditation with a different vocabulary, the words above may not mean the same thing to you as they do to me.

On sitting, I sit confortably but not too much.

On \textbf{observing} one's body, this is often referred to as body scan. I focus my attention on some part of my body, progressively changing my inner gaze (not my actual gaze) from part to part. I found that this is perhaps the skill that has evolved the most dramatically for me over the years, going from ``I'm looking at my hands'' to ``I am observing the sensation of hands'' to ``this is just a bunch of signals and I'm only making up the concept of hand''. A state of detachement that I would spend entire sessions to reach, I can now reach in a matter of seconds (not every time of course).

A number of meditation and mindfulness instructors make people count their breaths. By this act, one is effectively forced to balance between observing your breathing passively and actively breathing. The transition from one to the other is the transition between actor and observer. 

So what do I mean by observer? An observer acquires information passively, acknowledging its existence, its source, without presuming control over it. When one observes a bird in the wild, one has no control over it. When I truly observe myself, I am still my self, but I actively purse another point of view.

Once I'm in a state where I can confortably observe my body, I can start observing my \textbf{thoughts}. Just like with gaze, when looking around, one may be tempted to gaze at multiple things in rapid succession. When observing my body, I'm tempted to attend to various sensations, an itch, a pressure. When observing my thoughts, I'm tempted to follow new thoughts. Just like I can choose what to attend to when observing my body, I can choose to attend to a thought, or not. It really is the same mechanism.

When doing this choice, attending a thought or not, over and over, I usually eventually reach a state where the default arrival of new thoughts is not as something that interrupts, but rather as just another sensation. This is when I become the observer of my thoughts.

On \textbf{judgement}, this is perhaps one of the harder concepts to convey. You know when adults tell kids ``you'll understand one day'', it's that kind of concept. The first times I heard guided meditations telling me to observe without judgement, I was quite confused. Judgement usually appears as an irritation of being unable to focus, to attend, to reach a certain state. Perhaps one helpful way to think about judgement is as just another tought, which can be observed. I've found that after attending to judgement, it tends not to resurface. 

On \textbf{exercising} meditation daily, just like one would use weights to get stronger bisceps, I like to think of daily meditation as an exercise in attention for the brain. I don't think this analogy is a stretch, repeating mental processes really is like exercising, in that you get better at it.



\subsection{Some warnings/criticisms of the mindfulness-folklore}

On the C-ness-word, I don't like that word at all. Perhaps it is because of my study of machine learning and artificial neural networks, and my vague understanding of neuroscience. It is ill-defined at best, and harmful at worst. Using this word to teach meditation does not seem useful to me. Attention is something that's very well defined (in contrast) and that's easily communicable, because we have an immediate real-world metaphor for it, gaze.

On ego, self, and similar concepts, I think they can be useful to convey some sentiments. But be warned, in my (limited) experience meditation has little to do with the ``dissolution of the self'' as can be experienced with certain chemically-induced (e.g. with psilocybin) ego-dissolution experiences. I don't think it is useful or even healthy to promote conceptions of meditation which suggest this.

For example, I've heard it suggested that meditation can induce one to relativise away their emotions and problems, because, after all, all the problems we will ever have are only mental constructs arising from our Ego. While the latter is technically the truth, it is extremely unhelpful, and I'd argue harmful to someone learning meditation. You can't meditate your problems away, but you can practice paying \emph{attention}, observing your problems, rather than perhaps impulsively reacting.

\textbf{Mindfulness and Ideology}

To grossly oversimplify, mindfulness can help one accept their situation in the world.

A common criticism of such a stance is that we should not accept our situation. The world at large is full of problems, and opportunities to externally improve our wellbeings are many (think racism, sexism, etc). I often hear this criticism from people critical of the neoliberal ideology which permeates the West. I for one am critical of neoliberalism, and do agree that it almost exists as an axiom of our modern condition. In this sense, being accepting of our condition in the extreme means not being able to consider rejecting neoliberalism, or more generally, ideology (and I use ideology here in its Marxist/Althusser-ian sense)

Is this a fair criticism of mindfulness though? I see two failures to this argument. First, I think criticism (of neoliberalism) and acceptance of one's situation are not at odds. Emotional acceptance doesn't mean lack of emotion, nor lack of action; I can accept that I am hungry without lashing out at my pan and still work to get my ingredients cooked. Second, I would expect the clarity of mind that mindfulness can provide to allow one to be \emph{more} perceptive of their condition, not less. If I am constantly hungry and overtaken by hunger, I may spend all my time looking for food; if I can accept hunger and set it aside for a moment, I may be able to plan and get more nutritious food so that hunger becomes less of a problem in the future.



